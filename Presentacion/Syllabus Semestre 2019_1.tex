\documentclass[12pt]{article}
\usepackage[utf8]{inputenc}
\usepackage[spanish]{babel}
\usepackage[autostyle,spanish=mexican]{csquotes}
\usepackage{amsmath}
\usepackage{amsthm}
\usepackage{hyperref}
\usepackage{graphicx}
\usepackage{color}
\usepackage{float}
\usepackage{multicol}
\usepackage{enumerate}
\usepackage{anyfontsize}
\usepackage{anysize}
\usepackage{cite}
\renewcommand{\baselinestretch}{1.5}
\setlength{\parskip}{1em}
\marginsize{1.5cm}{1.5cm}{1cm}{2cm}
\author{M. en C. Gustavo Contreras Mayén. \texttt{curso.fisica.comp@gmail.com}\\
M. en C. Abraham Lima Buendía. \texttt{abraham3081@ciencias.unam.mx}}
\title{Curso de Física\\{\large Semestre 2019-1 Grupo 5378}}
\date{ }
\makeatletter
\renewcommand{\@biblabel}[1]{}
\renewenvironment{thebibliography}[1]
     {\section*{\refname}%
      \@mkboth{\MakeUppercase\refname}{\MakeUppercase\refname}%
      \list{}%
           {\labelwidth=0pt
            \labelsep=0pt
            \leftmargin1.5em
            \itemindent=-1.5em
            \advance\leftmargin\labelsep
            \@openbib@code
            }%
      \sloppy
      \clubpenalty4000
      \@clubpenalty \clubpenalty
      \widowpenalty4000%
      \sfcode`\.\@m}
\makeatother
\usepackage{breakcites}	
\begin{document}
%\renewcommand\theenumii{\arabic{theenumii.enumii}}
\renewcommand\labelenumii{\theenumi.{\arabic{enumii}}}
\maketitle
\fontsize{12}{12}\selectfont
\textbf{Lugar: }Laboratorio de Física General 1. Departamento de Física.

\textbf{Horario: } Martes y Viernes de 11 a 14 horas.

\textbf{Objetivos y Temario:} Se trabajará el temario oficial de la asignatura, que está disponible en: \href{http://www.fciencias.unam.mx/asignaturas/1102.pdf}{http://www.fciencias.unam.mx/asignaturas/1102.pdf}
\section{Metodología de Enseñanza.}
\textbf{Antes de la clase.} 

Para facilitar la discusión en el aula, el alumno revisará el material de trabajo que se le proporcionará oportunamente, de tal manera que llegará a la clase conociendo el tema a desarrollar. Daremos por entendido de que el alumno realizará la lectura y actividades establecidas.

\textbf{Durante la clase.}

Se dará un tiempo para la exposición con diálogo y discusión del material de trabajo con los temas a cubrir durante el semestre. Se realizarán prácticas experimentales en equipo, distribuidas a lo largo del semestre, con la finalidad de montar tanto equipo como material para reproducir un fenómeno físico y caracterizarlo.

\textbf{Después de la clase.}

El curso requiere que le dediquen al menos el mismo número de horas de trabajo en casa, es decir, les va a demandar seis horas como mínimo; tanto para resolver y entregar tareas semanales, como para el análisis de los datos obtenidos en las prácticas para elaborar un reporte en equipo.
\section{Evaluación.}
Los elementos y la proporción de la calificación del curso que se promediarán al final, se distribuyen de la siguiente manera:
\begin{itemize}
\item \textbf{Tareas semanales $\mathbf{50\%}$} : Con la finalidad de que el alumno desarrolle la capacidad de solución de problemas, se les proporcionará cada semana, un conjunto de ejercicios que deberá de entregar, la fecha de entrega quedará definida con anticipación, y se avisa que no se recibirán tareas extemporáneas. Las tareas son individuales y se entregarán a mano, es decir, no se recibirán por correo electrónico o memoria usb. Para que una tarea cuente, deberán de entregar al menos el $50\%$ de los ejercicios resueltos. 
\item \textbf{Exámenes parciales $\mathbf{30\%}$} : Habrá cuatro exámenes parciales en clase, que cubrirán los temas revisados.
\begin{enumerate}
\item Primer examen parcial: Tema 1. Conceptos básicos.
\item Segundo examen parcial: Tema 2. Electricidad y magnetismo.
\item Tercer examen parcial: Tema 3. Óptica.
\item Cuarto examen parcial: Tema 4. Estructura de la materia. 
\end{enumerate}
\item \textbf{Prácticas $\mathbf{20\%}$} : Por cada práctica que se realice, se entregará un reporte por mesa de trabajo, en donde se discutirán los resultados obtenidos, se proporcionará una guía para implementar cada práctica y una guía general para elaborar el reporte.
\end{itemize}
La calificación final se obtendrá de los porcentajes indicados para las tareas semanales, exámenes parciales y prácticas. En el caso de obtener un promedio final mayor o igual a $6$, será la calificación que se asentará en el acta, siendo la calificación definitiva que obtendría el alumno.
\section{Un examen de reposición.}
Considerando que sólo habrá cuatro exámenes durante el semestre, se considera la posibilidad de que el alumno presente una única reposición si y sólo si se cumplen cada uno de los siguientes puntos:
\begin{itemize}
\item El alumno presentó y entregó los cuatro exámenes parciales.
\item Sólo un examen parcial (de los cuatro) tenga una calificación no aprobatoria, es decir, que la calificación del examen parcial sea menor a $6$ (seis)
\item El alumno debió de haber realizado todas las prácticas y haber entregado los respectivos reportes.
\item El alumno debió de haber entregado todas las tareas semanales.
\end{itemize}
En caso de contar con un promedio final aprobatorio del curso (los cuatro exámenes parciales aprobados), no se aplicará una reposición de algún examen para subir el promedio final del curso.
\section{Examen final.}
El examen final del curso se presentará si y sólo si:
\begin{itemize}
\item El alumno presentó y entregó los cuatro exámenes parciales.
\item El alumno entregó todas las tareas del curso.
\item El alumno realizó todas las prácticas y entregó todos los reportes.
\item La calificación de dos exámenes parciales del alumno (o más exámenes parciales) tienen una calificación menor a seis.
\end{itemize}
En caso de presentar el examen final, la calificación obtenida, es la que se asentará en el acta de calificaciones del curso de Física. \textbf{Ya no se promediará con las tareas semanales ni con las prácticas}.
\par
\emph{En caso de haber presentado al menos un examen parcial y/o haber entregado al menos una tarea} se promediarán respectivamente las tareas, exámenes y prácticas.


Sólo se asentará en el acta de calificaciones \textcolor{blue}{NP} si el(la) alumn{@} no entrega tarea alguna y no presenta algún examen. (¿?)
\par
De acuerdo al Reglamento General de Exámenes de la UNAM, se considera una calificación aprobatoria aquella que sea mayor o igual a $6$ seis.
\begin{itemize}
\item No \enquote{se guardan calificaciones}.
\item No se renuncia a una calificación.
\end{itemize}
\section{Fechas importantes.}
\begin{itemize}
\item Lunes 6 de agosto. Inicio del semestre 2019-1.
\item Viernes 2 de noviembre. Día de Muertos - Feriado.
\item Viernes 23 de noviembre, Fin de Semestre.
\item Del 26 al 30 de noviembre, primera semana de finales.
\item Del 3 al 7 de diciembre,  segunda semana de finales.
\end{itemize}
\section{Bibliografía.}
Se recomienda la consulta de los siguientes textos que están disponibles en la biblioteca de la Facultad, en cada uno de los temas se propocionará bibliografía adicional para una mejor comprensión del tema.
\nocite{*}
\bibliographystyle{plain}
\bibliography{LibrosFisica2019}
\end{document}