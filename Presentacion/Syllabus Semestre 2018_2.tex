\documentclass[12pt]{article}
\usepackage[utf8]{inputenc}
\usepackage[spanish]{babel}
\usepackage[autostyle,spanish=mexican]{csquotes}
\usepackage{amsmath}
\usepackage{amsthm}
\usepackage{hyperref}
\usepackage{graphicx}
\usepackage{color}
\usepackage{float}
\usepackage{multicol}
\usepackage{enumerate}
\usepackage{anyfontsize}
\usepackage{anysize}
\usepackage{cite}
\renewcommand{\baselinestretch}{1.5}
\marginsize{1.5cm}{1.5cm}{1cm}{2cm}
\author{M. en C. Gustavo Contreras Mayén. \texttt{curso.fisica.comp@gmail.com}\\
M. en C. Abraham Lima Buendía. \texttt{abraham3081@ciencias.unam.mx}}
\title{Curso de Física Computacional\\{\large Semestre 2018-2 Grupo 8212}}
\date{ }
\makeatletter
\renewcommand{\@biblabel}[1]{}
\renewenvironment{thebibliography}[1]
     {\section*{\refname}%
      \@mkboth{\MakeUppercase\refname}{\MakeUppercase\refname}%
      \list{}%
           {\labelwidth=0pt
            \labelsep=0pt
            \leftmargin1.5em
            \itemindent=-1.5em
            \advance\leftmargin\labelsep
            \@openbib@code
            }%
      \sloppy
      \clubpenalty4000
      \@clubpenalty \clubpenalty
      \widowpenalty4000%
      \sfcode`\.\@m}
\makeatother
\usepackage{breakcites}	
\begin{document}
%\renewcommand\theenumii{\arabic{theenumii.enumii}}
\renewcommand\labelenumii{\theenumi.{\arabic{enumii}}}
\maketitle
\fontsize{12}{12}\selectfont
\textbf{Lugar: }Laboratorio de Enseñanza en Cómputo de Física, Edificio Tlahuizcalpan.
\\
\textbf{Horario: } Martes y Jueves de 18 a 21 horas.
\\
\\
\textbf{Objetivos y Temario:} Se trabajará el temario oficial de la asignatura, que está disponible en: \href{http://www.fciencias.unam.mx/asignaturas/715.pdf}{http://www.fciencias.unam.mx/asignaturas/715.pdf}
\section{Metodología de Enseñanza}
\textbf{Antes de la clase.}
\\
Para facilitar la discusión en el aula, el alumno revisará el material de trabajo que se le proporcionará oportunamente, de tal manera que llegará conociendo el tema a desarrollar a la clase. Daremos por entendido de que el alumno realizará la lectura y actividades establecidas.
\\
\textbf{Durante la clase.}
\\
Se dará un tiempo para la exposición con diálogo y discusión del material de trabajo con los temas a cubrir durante el semestre. Se busca que sea un curso totalmente práctico por lo que se va a trabajar con los equipos de cómputo del laboratorio.
\\
\textbf{Después de la clase.}
\\
El curso requiere que le dediquen al menos el mismo número de horas de trabajo en casa, es decir, les va a demandar seis horas como mínimo; si cuentan con una experiencia en programación, tienen un paso adelantado, pero si no han programado, se verán en la necesidad de dedicarle más tiempo.
\section{Evaluación}
Los elementos y la proporción de la calificación del curso, se distribuyen de la siguiente manera:
\begin{itemize}
\item \textbf{Ejercicios en clase $\mathbf{10\%}$:} para tener derecho a este porcentaje se requiere estar presente en la clase, es decir, el ejercicio se entregará en la clase o se dejará para la siguiente, en caso de que no asistan y se enteren del ejercicio, se les revisará el trabajo que entreguen, pero no se les tomará en cuenta para el porcentaje, (moraleja: hay que asistir a clase) 
\item \textbf{Tareas $\mathbf{50\%}$} : Serán seis tareas durante el curso, se les proporcionará de manera adelantada y con fecha de entrega definida, no se reciben tareas extemporáneas, ni por correo. Para evaluar las tareas, éstas deberán de contar con al menos el $50\%$ de los ejercicios resueltos. 
\item \textbf{Exámenes $\mathbf{40\%}$} : Habrá tres exámenes en clase, de tipo teórico-prácticos. 
\end{itemize}
La calificación final se obtendrá de los porcentajes indicados para los ejercicios en clase, de las tareas y de los exámenes. En el caso de obtener un promedio mayor o igual a $6$, es el que se asentará en el acta, que corresponde a la calificación definitiva del curso.
\section{Reposición}
Considerando que sólo habrá tres exámenes en el semestre, se considera la posibilidad de presentar una única reposición si y sólo si se cumplen los siguientes puntos:
\begin{itemize}
	\item Sólo un examen parcial tenga una calificación no aprobatoria, es decir, que la calificación del examen parcial sea menor a $6$ (seis)
	\item Se debieron de haber presentado los otros dos exámenes parciales.
	\item Se debieron de haber entregado las tres tareas completas.
\end{itemize}
En caso de contar con un promedio final aprobatorio del curso (los tres exámenes parciales aprobados), no se aplicará una reposición de algún examen para subir el promedio final del curso.
\section{Examen final}
El examen final del curso se presentará si y sólo si:
\begin{itemize}
\item Se presentaron los tres los exámenes parciales.
\item Entregaron las tres tareas del curso.
\item Hay dos exámenes parciales con calificación menor a seis.
\end{itemize}
La calificación obtenida en el examen final, es la que se asentará en el acta de calificaciones del curso de Física Computacional.
\\
\\
Ya no se promediará con las tareas ni con los ejercicios de clase.\\
\\
\emph{En caso de haber presentado al menos un examen parcial y/o haber entregado al menos una tarea} se promediarán respectivamente las tareas y exámenes.
\\
\\
Sólo se asentará en el acta de calificaciones \textcolor{blue}{NP} si el(la) alumn{@} no entrega tarea alguna y no presenta algún examen. (¿?)
\\
\\
De acuerdo al Reglamento General de Exámenes de la UNAM, se considera una calificación aprobatoria aquella que sea mayor o igual a $6$ seis.
\begin{itemize}
\item No \enquote{se guardan calificaciones}.
\item No se renuncia a una calificación.
\end{itemize}
\section{Fechas importantes}
\begin{itemize}
\item Lunes 29 de enero. Inicio del semestre 2018-2.
\item Lunes 26 al viernes 29 de marzo, Semana Santa.
\item Martes 1 de mayo, Día del Trabajo - Feriado.
\item Jueves 10 de mayo, Día de la Madre - Feriado.
\item Martes 15 de mayo, Día del Maestro - Feriado.
\item Viernes 29 de mayo. Fin de Semestre.
\item Del 28 de mayo al 1 de junio, primera semana de finales.
\item Del 4 al 8 de junio, segunda semana de finales.
\end{itemize}
\newpage
\section{Bibliografía.}
Se recomienda la consulta de los siguientes textos, en cada uno de los temas se propocionará bibliografía adicional para una mejor comprensión del tema.
\nocite{*}
\bibliographystyle{apalike-es}
\bibliography{LibrosFC}
\end{document}