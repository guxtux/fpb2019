%\RequirePackage[l2tabu, orthodox]{nag}
\documentclass[12pt]{beamer}
\graphicspath{{Imagenes/}{../Imagenes/}}
\usepackage[utf8]{inputenc}
\usepackage[spanish]{babel}
\usepackage[autostyle,spanish=mexican]{csquotes}
\usepackage{hyperref}
\hypersetup{
  colorlinks=true,
  linkcolor=blue,          % color of internal links (change box color with linkbordercolor)
  citecolor=green,        % color of links to bibliography
  filecolor=magenta,      % color of file links
  urlcolor=cyan,           % color of external links
  linkbordercolor={0 0 1}
}
\usepackage{amsmath}
\usepackage{amsthm}
\usepackage{amsfonts}
\usepackage{multicol}
\usepackage{graphicx}
\usepackage{tabulary}
\usepackage{booktabs}
\usepackage{epstopdf}
\usepackage{media9}
\usepackage{cancel}
\usepackage[binary-units=true]{siunitx}
\usepackage{standalone}
\usepackage{longtable}
\usepackage{bigints}
\usepackage{caption}
%\usepackage{enumitem}
\usepackage{tikz}
\usetikzlibrary{mindmap}
\usepackage[siunitx]{circuitikz}
\usetikzlibrary{arrows, patterns, shapes, decorations.markings}
\usetikzlibrary{matrix,positioning}
\tikzstyle{every picture}+=[remember picture,baseline]
\usepackage{color}
\usepackage{alltt}
\usepackage{verbatim}

\usepackage{fancyvrb}
\usepackage[os=win]{menukeys}
\usepackage{pifont}
\usepackage[sfdefault]{roboto}  %% Option 'sfdefault' only if the base font of the document is to be sans serif
%\usepackage[T1]{fontenc}
\setcounter{secnumdepth}{3}
\setcounter{tocdepth}{3}
\DeclareGraphicsExtensions{.pdf,.png,.jpg}
\renewcommand {\arraystretch}{1.5}
\definecolor{ao}{rgb}{0.0, 0.5, 0.0}
\definecolor{bisque}{rgb}{1.0, 0.89, 0.77}
\definecolor{amber}{rgb}{1.0, 0.75, 0.0}
\definecolor{armygreen}{rgb}{0.29, 0.33, 0.13}
\definecolor{alizarin}{rgb}{0.82, 0.1, 0.26}
\definecolor{cadetblue}{rgb}{0.37, 0.62, 0.63}
\newcommand*{\TitleParbox}[1]{\parbox[c]{6cm}{\raggedright #1}}%
\newcommand{\python}{\texttt{python}}
\newcommand{\textoazul}[1]{\textcolor{blue}{#1}}
\newcommand{\azulfuerte}[1]{\textcolor{blue}{\textbf{#1}}}
\newcommand{\funcionazul}[1]{\textcolor{blue}{\textbf{\texttt{#1}}}}
%\normalfont
\usepackage{ccfonts}% http://ctan.org/pkg/{ccfonts}
\usepackage[T1]{fontenc}% http://ctan.or/pkg/fontenc
\renewcommand{\rmdefault}{cmr}% cmr = Computer Modern Roman
\usefonttheme[onlymath]{serif}
\linespread{1.3}
\newcounter{saveenumi}
\newcommand{\seti}{\setcounter{saveenumi}{\value{enumi}}}
\newcommand{\conti}{\setcounter{enumi}{\value{saveenumi}}}

%reduce el tamaño de letra de la etiqueta equations
\makeatletter
\def\maketag@@@#1{\hbox{\m@th\normalfont\small#1}}
\makeatother

%se usa para la x en itemize
\newcommand{\xmark}{\text{\ding{55}}}

%\AtBeginDocument{\setlength{\tymin}{1em}}


\definecolor{myblue}{rgb}{.8, .8, 1}

\usepackage{amsmath}
\usepackage{empheq}

\newlength\mytemplen
\newsavebox\mytempbox

\makeatletter
\newcommand\mybluebox{%
    \@ifnextchar[%]
       {\@mybluebox}%
       {\@mybluebox[0pt]}}

\def\@mybluebox[#1]{%
    \@ifnextchar[%]
       {\@@mybluebox[#1]}%
       {\@@mybluebox[#1][0pt]}}

\def\@@mybluebox[#1][#2]#3{
    \sbox\mytempbox{#3}%
    \mytemplen\ht\mytempbox
    \advance\mytemplen #1\relax
    \ht\mytempbox\mytemplen
    \mytemplen\dp\mytempbox
    \advance\mytemplen #2\relax
    \dp\mytempbox\mytemplen
    \colorbox{myblue}{\hspace{1em}\usebox{\mytempbox}\hspace{1em}}}

\makeatother

\sisetup{separate-uncertainty}%


%Se usa la plantilla Warsaw modificada con spruce
\mode<presentation>
{
  \usetheme{Warsaw}
  \setbeamertemplate{headline}{}
  \useoutertheme{default}
  %\usecolortheme{beaver}
  \setbeamercovered{invisible}
}
%\AtBeginSection[]
%{
%\begin{frame}<beamer>{Contenido}
%\normalfont\mdseries
%\tableofcontents[currentsection]
%\end{frame}
%}

\setbeamertemplate{section in toc}[sections numbered]
\setbeamertemplate{subsection in toc}[subsections numbered]
\setbeamertemplate{subsection in toc}{\leavevmode\leftskip=3.2em\rlap{\hskip-2em\inserttocsectionnumber.\inserttocsubsectionnumber}\inserttocsubsection\par}
\setbeamercolor{section in toc}{fg=blue}
\setbeamercolor{subsection in toc}{fg=blue}
\setbeamertemplate{navigation symbols}{}
\setbeamercolor{frametitle}{fg=yellow,bg=blue!70!white}
\setbeamercolor{section in head/foot}{bg=gray!30,fg=red}
%\setbeamercolor{section in head}{bg=green,fg=red}
\setbeamercolor{subsection in head/foot}{bg=gray!30,fg=black}
\setbeamercolor{author in head/foot}{bg=gray!30}
\setbeamercolor{date in head/foot}{fg=blue}

%\mode<presentation>
%{
%  \usetheme{Warsaw}
%  \setbeamertemplate{headline}{}
%  %\useoutertheme{infolines}
%  \useoutertheme{default}
%  \setbeamercovered{invisible}
%  % or whatever (possibly just delete it)
%}

%\input{../Preambulos/pre_codigo}
\makeatletter
\setbeamertemplate{footline}
{
  \leavevmode%
  \hbox{%
  \begin{beamercolorbox}[wd=.333333\paperwidth,ht=2.25ex,dp=1ex,center]{author in head/foot}%
    \usebeamerfont{author in head/foot} \insertsection
  \end{beamercolorbox}}%
  \begin{beamercolorbox}[wd=.333333\paperwidth,ht=2.25ex,dp=1ex,center]{title in head/foot}%
    \usebeamerfont{title in head/foot} \insertsubsection
  \end{beamercolorbox}%
  \begin{beamercolorbox}[wd=.333333\paperwidth,ht=2.25ex,dp=1ex,right]{date in head/foot}%
    \usebeamerfont{date in head/foot} \textcolor{white}{\insertshortdate{}} \hspace*{2em}
    \textcolor{white}{\insertframenumber{} / \inserttotalframenumber}\hspace*{2ex} 
  \end{beamercolorbox}}%
  \vskip0pt%
\makeatother
\title{\large{Tema 1 - Conceptos básicos}}
\subtitle{Curso de Física}
\author[]{M. en C. Gustavo Contreras Mayén}
\date{\today}
\institute{Facultad de Ciencias - UNAM}
\titlegraphic{\includegraphics[width=2cm]{../Imagenes/escudo-facultad-ciencias}\hspace*{4.75cm}~%
   \includegraphics[width=2cm]{../Imagenes/escudo-unam}
}
\begin{document}
\maketitle
\section*{Contenido}
\frame[allowframebreaks]{\tableofcontents[currentsection, hideallsubsections]}
\fontsize{14}{14}\selectfont
\spanishdecimal{.}
\section{¿Qué estudia la física?}
\frame{\tableofcontents[currentsection, hideothersubsections]}
\subsection{Introducción}
\begin{frame}[plain]
\begin{figure}
\includestandalone[scale=0.55]{Figuras/mapamental2}
\end{figure}
\end{frame}
\begin{frame}   
\frametitle{¿Qué estudia la física?}
La física estudia lo grande y lo pequeño, lo viejo y lo nuevo. 
\\
\bigskip
Del átomo a las galaxias, de los circuitos eléctricos a la aerodinámica, la física es una gran parte del mundo que nos rodea.
\end{frame}
\begin{frame}
\frametitle{Reflexión}
Deviene claro que el ulterior progreso en cualquier ciencia es imposible sin la utilización de los logros de otras ramas del conocimiento.
\end{frame}
\begin{frame}
\frametitle{Reflexión}
La matemática y la física. La física y la química. La matemática y la electrónica.
\\
\bigskip
\pause
La simbiosis de estas ciencias exactas, actualmente parece natural, y la física matemática, la química física y la matemática de computación surgidas como resultado de esta simbiosis ya hace mucho que se han convertido en nombres acostumbrados.
\end{frame}
\begin{frame}
\frametitle{Reflexión}
Quiso la suerte que la biología y la medicina no fueran a parar a la categoría de ciencias exactas.
\\
\bigskip
El objeto de estudio de estas ciencias, el organismo vivo, es hasta tal grado complejo y multiforme que no siquiera hoy en día existe la posibilidad de describir con precisión todas sus características y regularidades.
\end{frame}
\begin{frame}
\frametitle{Reflexión}
En el curso de muchos siglos la biología intervenía tan sólo como una ciencia descriptiva y, prácticamente, no explicaba las causas de la mayoría de los fenómenos que transcurren en el organismo vivo.
\end{frame}
\begin{frame}
\frametitle{Reflexión}
La utilización de los logros de la física y de la química ofreció la posibilidad de investigar los fundamentos de la vida a nivel molecular.
\end{frame}
\begin{frame}
\frametitle{Reflexión}
Como resultado de la interpretación de la química y la biología, así como de la física y la biología, se originaron la bioquímica y la biofísica.
\pause
\vfill
K. Bogdánov. El Físico visita al biólogo. Editorial Mir (1986)
\end{frame}
\begin{frame}
\frametitle{La física en el estudio}
El estudio de la física es también una aventura:
\pause
\setbeamercolor{item projected}{bg=blue!70!black,fg=yellow}
\setbeamertemplate{enumerate items}[circle]
\begin{enumerate}[<+->]
\item A veces frustante.
\item En otras, dolorosa.
\end{enumerate}
\end{frame}
\begin{frame}
\frametitle{Como hacer física}
La física es una ciencia experimental.
\\
\bigskip
A través de la observación de fenómenos naturales, se busca determinar un modelo y los principios que los describen.
\end{frame}
\begin{frame}   
\frametitle{Modelos en física}
Esos modelos se conocen como teorías físicas, si están muy bien establecidos y se usan ampliamente, se le llaman leyes.
\\
\bigskip
\end{frame}
\section{Medición e incertidumbre}
\begin{frame}

\end{frame}
\section{Cinemática}
\begin{frame}

\end{frame}


\end{document}