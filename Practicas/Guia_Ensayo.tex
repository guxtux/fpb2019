\documentclass[12pt]{article}
\usepackage[utf8]{inputenc}
\usepackage[spanish,mexico]{babel}
%\usepackage[showframe, letterpaper]{geometry}
\usepackage[letterpaper]{geometry}
\geometry{top=1.25cm, bottom=1.0cm, left=2cm, right=2cm}
\usepackage{amsmath}
\usepackage{amsthm}
\usepackage{enumitem}
\setenumerate[1]{label=\thesection.\arabic*.}
\setenumerate[2]{label*=\arabic*.}
\usepackage{array}
\newcolumntype{C}[1]{>{\centering\arraybackslash}p{#1}}
\usepackage{caption}
\usepackage{titlesec}
%\titlespacing\section{0pt}{12pt plus 4pt minus 2pt}{0pt plus 2pt minus 2pt}
\titlespacing{\section}{0pt}{\parskip}{-\parskip}
\usepackage{multicol,multienum}
\usepackage{graphicx}
\usepackage{standalone}
\usepackage[outdir=../]{epstopdf}
\usepackage[binary-units=true]{siunitx}
\usepackage{float}
\DeclareGraphicsExtensions{.pdf,.png,.jpg}
\usepackage{tikz}
\usetikzlibrary{patterns}
\usetikzlibrary{decorations.pathmorphing,patterns}
\usetikzlibrary{arrows,calc,patterns,decorations.markings}
\usetikzlibrary{positioning}
\usepackage{color}
\usepackage{anysize}
\usepackage[spanish=mexican]{csquotes}
\usepackage{anyfontsize}
\usepackage[os=win]{menukeys}
\usepackage{pbox}
%Este paquete permite manejar los encabezados del documento
%\usepackage{fancyhdr}
%hay que definir el ambiente de la página
%\pagestyle{fancy}
%\lfoot{\small{Material elaborado por: M. en C. Gustavo Contreras Mayén \\ \hspace{4.3cm} M. en C. Abraham Lima Buendía.}}

%aqui va el texto para todas las paginas l--> izquierda, r--> derecha, hay un C--> para centrar el texto deseado
%\lhead{Curso de Física Computacional}
%\fancyhead[R]{\nouppercase{\leftmark}}
%define el ancho de la linea que separa el encabezado del cuerpo del texto
%\renewcommand{\headrulewidth}{0.5pt}
\setlength{\parskip}{1em}
\renewcommand{\baselinestretch}{1.25}
\newcommand{\python}{\texttt{python}}
\newcommand{\funcionazul}[1]{\textcolor{blue}{\textbf{\texttt{#1}}}}
\interfootnotelinepenalty=8000
\usepackage{hyperref}
%esta parte define el color del marco que aparece en las hiperreferencias.
\definecolor{links}{HTML}{2A1B81}
\hypersetup{colorlinks,linkcolor=,urlcolor=links}
\spanishdecimal{.}
\marginsize{1.5cm}{1.5cm}{1.5cm}{1.5cm}
\numberwithin{equation}{section}
\date{}
\usepackage[sfdefault]{roboto}  %% Option 'sfdefault' only if the base font of the document is to be sans serif
\usepackage[T1]{fontenc}
\title{Guía para la entrega de ensayos \\ \begin{Large}Curso de Física 2019-2\end{Large}}
%\author{M. en C. Gustavo Contreras Mayén.}
\setlength{\voffset}{-1cm}
\begin{document}
\maketitle
\vspace*{-2cm} 
\fontsize{14}{14}\selectfont
Los ensayos tienen como finalidad que el alumno extienda el estudio de un tema en particular del curso, para ello deberá de preparar un escrito con una serie de apartados. La extensión del ensayo es de 5 páginas.
\par
La redacción del texto será un trabajo individual, permitiendo el desarrollo de habilidades de razonamiento, abstracción, pensamiento crítico y de expresión por parte del alumno.
\par
La hoja de ensayo que se entregue en la clase, contendrá la principal referencia de consulta: una página web, un video, un artículo, un capítulo de libro, etc. así como una serie de preguntas orientadoras que servirán de estructura para la redacción del ensayo.
\par
A continuación se señalan los apartados que deben de incluirse de manera obligatoria en el escrito.
\section{Nombre.}
Se debe de anotar el nombre completo del alumno, así como su número de cuenta. No se tomarán en cuenta aquellos ensayos colectivos.
\section{Introducción.}
Se debe de incluir una breve introducción general al tema, considerando el alcance o limitaciones (no se busca una exposición extensa del tema, sino que el alumno extienda por su cuenta el mismo); así como en el reporte de las prácticas, la \textbf{Introducción }debe de estar fundamentada con una bibliogragía.
\section{Desarrollo.}
Como se mencionó previamente, habrá una serie de preguntas orientadoras que facilitarán el estudio del tema, se espera que el alumno logre responder cada una de ellas luego de haber hecho la revisión de la fuente u objeto de estudio.
\par
Cuenta el alumno con la libertad de incluir cada una de las preguntas y responderlas, de esta manera estaría desarrollando el cuerpo principal del ensayo, pero lo que esperamos es que logre una redacción más elaborada apoyándose con esas preguntas orientadoras, de ser posible, concatenando sus ideas con las respuestas a las preguntas.
\par
El alumno deberá de exponer en este apartado sus opiniones, críticas, comentarios sobre el tema, argumentando con lo que revisó en la introducción.
\par
Puede contrastar tanto hipótesis, ideas, teorías de diferentes escuelas de pensamiento, así como de autores que alimenten la discusión y opinión al tema.
\section{Conclusiones.}
Este apartado considera la relación entre lo que menciona en la \textbf{Introducción} y el \textbf{Desarrollo}, de tal manera que exprese de manera ordenada los puntos relevantes que encontró en la revisión del tema.
\par
En las conclusiones se retoma el tema de estudio (o fenómeno), para así resaltar la importancia de los argumentos expuestos por el alumno, y que sustentan las respuestas a las preguntas orientadoras.
\section{Bibliografía.}
Se deben de incluir todas las fuentes de información consultadas, ya sean revistas científicas, libros, sitios web (No Wikipedia, etc.)
\par
Con el constante uso de referencias, el alumno logrará la habilidad de incluir en su texto, las citas de una manera profesional, que en el futuro le será de gran apoyo tanto para la escritura de una tesis, como de la elaboración de artículos.
\vfill
\small{Material elaborado por: M. en C. Gustavo Contreras Mayén. \\ \hspace*{4.57cm} M. en C. Abraham Lima Buendía.}
\end{document}