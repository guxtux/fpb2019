\documentclass[12pt]{article}
\usepackage[utf8]{inputenc}
\usepackage[spanish,mexico]{babel}
%\usepackage[showframe, letterpaper]{geometry}
\usepackage[letterpaper]{geometry}
\geometry{top=1.25cm, bottom=1.0cm, left=2cm, right=2cm}
\usepackage{amsmath}
\usepackage{amsthm}
\usepackage{enumitem}
\setenumerate[1]{label=\thesection.\arabic*.}
\setenumerate[2]{label*=\arabic*.}
\usepackage{array}
\newcolumntype{C}[1]{>{\centering\arraybackslash}p{#1}}
\usepackage{caption}
\usepackage{titlesec}
%\titlespacing\section{0pt}{12pt plus 4pt minus 2pt}{0pt plus 2pt minus 2pt}
\titlespacing{\section}{0pt}{\parskip}{-\parskip}
\usepackage{multicol,multienum}
\usepackage{graphicx}
\usepackage{standalone}
\usepackage[outdir=../]{epstopdf}
\usepackage[binary-units=true]{siunitx}
\usepackage{float}
\DeclareGraphicsExtensions{.pdf,.png,.jpg}
\usepackage{tikz}
\usetikzlibrary{patterns}
\usetikzlibrary{decorations.pathmorphing,patterns}
\usetikzlibrary{arrows,calc,patterns,decorations.markings}
\usetikzlibrary{positioning}
\usepackage{color}
\usepackage{anysize}
\usepackage[spanish=mexican]{csquotes}
\usepackage{anyfontsize}
\usepackage[os=win]{menukeys}
\usepackage{pbox}
%Este paquete permite manejar los encabezados del documento
%\usepackage{fancyhdr}
%hay que definir el ambiente de la página
%\pagestyle{fancy}
%\lfoot{\small{Material elaborado por: M. en C. Gustavo Contreras Mayén \\ \hspace{4.3cm} M. en C. Abraham Lima Buendía.}}

%aqui va el texto para todas las paginas l--> izquierda, r--> derecha, hay un C--> para centrar el texto deseado
%\lhead{Curso de Física Computacional}
%\fancyhead[R]{\nouppercase{\leftmark}}
%define el ancho de la linea que separa el encabezado del cuerpo del texto
%\renewcommand{\headrulewidth}{0.5pt}
\setlength{\parskip}{1em}
\renewcommand{\baselinestretch}{1.25}
\newcommand{\python}{\texttt{python}}
\newcommand{\funcionazul}[1]{\textcolor{blue}{\textbf{\texttt{#1}}}}
\interfootnotelinepenalty=8000
\usepackage{hyperref}
%esta parte define el color del marco que aparece en las hiperreferencias.
\definecolor{links}{HTML}{2A1B81}
\hypersetup{colorlinks,linkcolor=,urlcolor=links}
\spanishdecimal{.}
\marginsize{1.5cm}{1.5cm}{1.5cm}{1.5cm}
\numberwithin{equation}{section}
\date{}
\usepackage[sfdefault]{roboto}  %% Option 'sfdefault' only if the base font of the document is to be sans serif
\usepackage[T1]{fontenc}
\title{Guía para las prácticas \\ \begin{Large}Curso de Física 2019-2\end{Large}}
%\author{M. en C. Gustavo Contreras Mayén.}
\setlength{\voffset}{-1cm}
\begin{document}
\maketitle
\vspace*{-2cm} 
\fontsize{14}{14}\selectfont
Esta guía para las prácticas contiene los puntos de información necesarios para el estudio previo del tema de la práctica, el montaje necesario para armar dentro del laboratorio, así como los puntos de discusión y análisis que deberán de considerar e incluir en el reporte.
\section{Objetivos.}
En este apartado se mencionan el(los) objetivo(s) a cumplir con el desarrollo de la práctica. Siempre serán objetivos logrables, por lo tanto, son medibles y están relacionados directamente con el apartado de Análisis experimental.
\par
Cada una de las prácticas es material que puede incluirse en el examen parcial de conocimientos.
\section{Conocimientos necesarios y referencias.}
En la sesión de práctica se tendrá el tiempo necesario para el montaje y desarrollo de la misma, por lo tanto, el alumno deberá de revisar los conocimientos necesarios y las referencias que se indiquen en este apartado, con la finalidad de que conozcan de antemano, el tema y el fenómeno que se va a estudiar.
\par
Durante el desarrollo de la práctica, el equipo docente revisará el desarrollo en cada mesa de trabajo, realizando preguntas a los alumnos elegidos al azar, por lo que se recomienda que estudien antes de la sesión. Se entregará el formato de práctica una clase antes de la misma.
\section{Material.}
Para el respectivo montaje de la práctica, se señalarán los materiales, elementos, dispositivos, equipos, medidores, etc. necesarios para la práctica.
\par
Para el préstamo de los materiales se debe de preparar un vale  y entregar una credencial al al laboratorista. Todo el equipo de la mesa es el responsable de la integridad de los materiales solicitados.
\par
\underline{Nota importante:} \textbf{Todo equipo y medidores eléctricos/electrónicos deberán de ser supervisados por los profesores antes de su uso.}
\par
Para contar con el suficiente tiempo de entrega de los materiales y equipo, se deberá de desmontar la práctica 15 minutos antes de que concluya la clase. Dejando organizada y limpia la mesa de trabajo.
\section{Procedimiento experimental.}
Una vez que ya se cuenta con el material para la práctica, todo el equipo de la mesa deberá de participar en el montaje de la práctica.
\par
Se señalará en este apartado el procedimiento que deberán de realizar, la metodología de trabajo, así como propuestas para la recolección de los datos. En algunas prácticas se realizarán mediciones con cambios en ciertos parámetros, por lo que deben de garantizar que cuentan con las mediciones y datos necesarios para su posterior análisis.
\par
En caso de que no se logre la recolección necesaria de la información, se menciona que \emph{no se podrá repetir la práctica en otro momento (es decir, no se podrá realizar la práctica en otra sesión o en otro horario)}.
\section{Análisis experimental.}
En este apartado se indicarán una serie de enunciados y preguntas orientadoras para el desarrollo, así como para la recolección de datos durante la práctica.
\par
No debe de considerarse como lo único y evidente para la práctica, es decir, pueden extender (tomando en cuenta el tiempo) más allá de los puntos mencionados. Se espera que cada uno de los puntos se resuelva satisfactoriamente y que se incluyan además en el reporte.
\vfill
\small{Material elaborado por: M. en C. Gustavo Contreras Mayén. \\ \hspace*{4.57cm} M. en C. Abraham Lima Buendía.}
\end{document}