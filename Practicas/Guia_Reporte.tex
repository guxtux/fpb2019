\documentclass[12pt]{article}
\usepackage[utf8]{inputenc}
\usepackage[spanish,mexico]{babel}
%\usepackage[showframe, letterpaper]{geometry}
\usepackage[letterpaper]{geometry}
\geometry{top=1.25cm, bottom=1.0cm, left=2cm, right=2cm}
\usepackage{amsmath}
\usepackage{amsthm}
\usepackage{enumitem}
\setenumerate[1]{label=\thesection.\arabic*.}
\setenumerate[2]{label*=\arabic*.}
\usepackage{array}
\newcolumntype{C}[1]{>{\centering\arraybackslash}p{#1}}
\usepackage{caption}
\usepackage{titlesec}
%\titlespacing\section{0pt}{12pt plus 4pt minus 2pt}{0pt plus 2pt minus 2pt}
\titlespacing{\section}{0pt}{\parskip}{-\parskip}
\usepackage{multicol,multienum}
\usepackage{graphicx}
\usepackage{standalone}
\usepackage[outdir=../]{epstopdf}
\usepackage[binary-units=true]{siunitx}
\usepackage{float}
\DeclareGraphicsExtensions{.pdf,.png,.jpg}
\usepackage{tikz}
\usetikzlibrary{patterns}
\usetikzlibrary{decorations.pathmorphing,patterns}
\usetikzlibrary{arrows,calc,patterns,decorations.markings}
\usetikzlibrary{positioning}
\usepackage{color}
\usepackage{anysize}
\usepackage[spanish=mexican]{csquotes}
\usepackage{anyfontsize}
\usepackage[os=win]{menukeys}
\usepackage{pbox}
%Este paquete permite manejar los encabezados del documento
%\usepackage{fancyhdr}
%hay que definir el ambiente de la página
%\pagestyle{fancy}
%\lfoot{\small{Material elaborado por: M. en C. Gustavo Contreras Mayén \\ \hspace{4.3cm} M. en C. Abraham Lima Buendía.}}

%aqui va el texto para todas las paginas l--> izquierda, r--> derecha, hay un C--> para centrar el texto deseado
%\lhead{Curso de Física Computacional}
%\fancyhead[R]{\nouppercase{\leftmark}}
%define el ancho de la linea que separa el encabezado del cuerpo del texto
%\renewcommand{\headrulewidth}{0.5pt}
\setlength{\parskip}{1em}
\renewcommand{\baselinestretch}{1.25}
\newcommand{\python}{\texttt{python}}
\newcommand{\funcionazul}[1]{\textcolor{blue}{\textbf{\texttt{#1}}}}
\interfootnotelinepenalty=8000
\usepackage{hyperref}
%esta parte define el color del marco que aparece en las hiperreferencias.
\definecolor{links}{HTML}{2A1B81}
\hypersetup{colorlinks,linkcolor=,urlcolor=links}
\spanishdecimal{.}
\marginsize{1.5cm}{1.5cm}{1.5cm}{1.5cm}
\numberwithin{equation}{section}
\date{}
\usepackage[sfdefault]{roboto}  %% Option 'sfdefault' only if the base font of the document is to be sans serif
\usepackage[T1]{fontenc}
\title{Guía para el reporte de práctica \\ \begin{Large}Curso de Física 2019-2\end{Large}}
%\author{M. en C. Gustavo Contreras Mayén.}
\setlength{\voffset}{-1cm}
\begin{document}
\maketitle
\vspace*{-2cm} 
\fontsize{14}{14}\selectfont
Luego de concluir una práctica, los integrantes de la mesa de trabajo deberán de reunirse para discutir y preparar un reporte colectivo, las secciones que se indican a continuación son de carácter obligatorio.
\par
Se estima que la coordinación y organización de trabajo fuera de clase se hará sin contratiempos. Debe de evitarse la división y asignación por secciones del reporte a los integrantes del equipo, para la unión final de aquéllos y considerar que se tiene un reporte. Se les otorgará el tiempo suficiente para la preparación de un buen reporte.
\section{Integrantes.}
Se deberá de incluir el nombre completo y número de cuenta de cada uno de los integrantes de la mesa.
\par
Los equipos de trabajo se mantendrán a lo largo del semestre, en caso de que un alumno no se haya presentado a la sesión de la práctica, no deberá de incluirse en el reporte, en caso contrario, se invalida la entrega del mismo, resultando en un \enquote{hueco} en la lista de evaluaciones (equivale a un cero), por lo que esta calificación promedia.
\section{Objetivo(s).}
Deberán de incluirse el(los) objetivo(s) tal cual se señalaron en la hoja de la práctica.
\section{Marco teórico.}
A manera de introducción al fenómeno en estudio, se debe de incluir un breve marco teórico (por breve señalamos que no debe de exceder de página y media\footnote{Considerando un reporte impreso en hoja tamaño carta, con tipografía de 12 puntos y renglón y medio de separación.}).
\par
Este marco teórico favorecerá la discusión en el apartado de \textbf{Resultados}, así como en las \textbf{Conclusiones} del reporte; lo que se incluya en este apartado, debe estar apoyado con la consulta de bibliografía que se mencionará en el respectivo apartado.
\section{Procedimiento experimental.}
Será una relatoría sobre lo que realizaron para la implementación de la práctica, omitiendo la parte de solicitar el material, sino más bien, enfocándose a la parte de preparación, calibración, y revisión del sistema o montaje: si se presentaron complicaciones, de cómo se resolvieron, etc.
\par
En el caso de que la práctica requiera una nueva medición luego de modificar un parámetro, no se requiere que se explique nuevamente el procedimiento, sino señalar el cambio que se hizo.
\par
Se debe de reportar en la medida de lo posible, todas aquellas situaciones que pudieron haber modificado las mediciones o registros: corrientes de aire que no se lograron desviar, pila baja en los medidores, vibraciones en la mesa, etc.
\par
Este apartado debe de considerar cada una de las preguntas o enunciados señalados en el apartado de \textbf{Análisis experimental} de la guía de la práctica.
\section{Resultados.}
Buena parte de la práctica consiste en la recolección de datos a través de observaciones, se deberá de implementar una estrategia lo más sencilla para ello. Se sugiere el uso de tablas de datos, por lo que deberán de estar identificadas todas las variables que se manejan.
\par
Cada registro anotado deberá de tener asociada la medida de incertidumbre, de tal manera que se operen debidamente estas cantidades.
\par
Se recomienda la inclusión de tablas y gráficas, que deberán de estar organizadas y numeradas para una lectura más sencilla.
\par
Luego de haber presentado el conjunto de datos experimentales, se procede a la discusión de los mismos con el marco teórico previamente incluido, es aquí en donde se debe de establecer de manera natural la relación entre el experimento con alguna expresión matemática que modele el fenómeno, señalando la correspondiente medida de incertidumbre, que nos determinará si nuestros datos experimentales guardan cierto comportamiento, a reserva de fuentes de errores sistemáticos, más no de errores aleatorios.
\section{Conclusiones.}
En este apartado se debe de señalar si el(los) objetivo(s) de la práctica se han cubierto, mencionando las diferencias, sus posibles fuentes de error y las correcciones necesarias para minimizarlas.
\par
Se espera que el grupo establezca la conexión entre el conjunto de datos experimentales con el fenómeno en estudio.
\section{Bibliografía.}
Deberán de incluir la bibliografía consultada para el reporte, tomen en cuenta de que ésta no necesariamente es exclusiva del apartado del Marco teórico, sino podrán apoyarse para reforzar su discusión y conclusiones en el reporte.
\par
Comentamos que existen hoy en día herramientas que permiten identificar en automático cuando un texto ha sido plagiado (\emph{técnica del copy-paste}), si ésta situación se presenta en el reporte, queda invalidado de manera automática.
\par
Eviten la inclusión de fuentes de consulta tipo: Wikipedia, El Rincón del Vago, etc., este tipo de sitios son buenos para una primera consulta que les servirá para establecer el contexto, pero no son los que deben de aparecer en un reporte de licenciatura.
\vfill
\small{Material elaborado por: M. en C. Gustavo Contreras Mayén. \\ \hspace*{4.57cm} M. en C. Abraham Lima Buendía.}
\end{document}