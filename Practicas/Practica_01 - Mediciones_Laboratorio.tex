\documentclass[12pt]{article}
\usepackage[utf8]{inputenc}
\usepackage[spanish,mexico]{babel}
%\usepackage[showframe, letterpaper]{geometry}
\usepackage[letterpaper]{geometry}
\geometry{top=1.25cm, bottom=1.0cm, left=2cm, right=2cm}
\usepackage{amsmath}
\usepackage{amsthm}
\usepackage{enumitem}
\setenumerate[1]{label=\thesection.\arabic*.}
\setenumerate[2]{label*=\arabic*.}
\usepackage{array}
\newcolumntype{C}[1]{>{\centering\arraybackslash}p{#1}}
\usepackage{caption}
\usepackage{titlesec}
%\titlespacing\section{0pt}{12pt plus 4pt minus 2pt}{0pt plus 2pt minus 2pt}
\titlespacing{\section}{0pt}{\parskip}{-\parskip}
\usepackage{multicol,multienum}
\usepackage{graphicx}
\usepackage{standalone}
\usepackage[outdir=../]{epstopdf}
\usepackage[binary-units=true]{siunitx}
\usepackage{float}
\DeclareGraphicsExtensions{.pdf,.png,.jpg}
\usepackage{tikz}
\usetikzlibrary{patterns}
\usetikzlibrary{decorations.pathmorphing,patterns}
\usetikzlibrary{arrows,calc,patterns,decorations.markings}
\usetikzlibrary{positioning}
\usepackage{color}
\usepackage{anysize}
\usepackage[spanish=mexican]{csquotes}
\usepackage{anyfontsize}
\usepackage[os=win]{menukeys}
\usepackage{pbox}
%Este paquete permite manejar los encabezados del documento
%\usepackage{fancyhdr}
%hay que definir el ambiente de la página
%\pagestyle{fancy}
%\lfoot{\small{Material elaborado por: M. en C. Gustavo Contreras Mayén \\ \hspace{4.3cm} M. en C. Abraham Lima Buendía.}}

%aqui va el texto para todas las paginas l--> izquierda, r--> derecha, hay un C--> para centrar el texto deseado
%\lhead{Curso de Física Computacional}
%\fancyhead[R]{\nouppercase{\leftmark}}
%define el ancho de la linea que separa el encabezado del cuerpo del texto
%\renewcommand{\headrulewidth}{0.5pt}
\setlength{\parskip}{1em}
\renewcommand{\baselinestretch}{1.25}
\newcommand{\python}{\texttt{python}}
\newcommand{\funcionazul}[1]{\textcolor{blue}{\textbf{\texttt{#1}}}}
\interfootnotelinepenalty=8000
\usepackage{hyperref}
%esta parte define el color del marco que aparece en las hiperreferencias.
\definecolor{links}{HTML}{2A1B81}
\hypersetup{colorlinks,linkcolor=,urlcolor=links}
\spanishdecimal{.}
\marginsize{1.5cm}{1.5cm}{1.5cm}{1.5cm}
\numberwithin{equation}{section}
\date{}
\title{Práctica 1 - Mediciones en el laboratorio \\ \begin{Large}Curso de Física 2019-1\end{Large}}
%\author{M. en C. Gustavo Contreras Mayén.}
\setlength{\voffset}{-1cm}
\begin{document}
\maketitle
\vspace*{-3cm} 
\fontsize{14}{14}\selectfont
\section{Objetivo:}
\begin{enumerate}
\item Expresar la incertidumbre de mediciones directa e indirectas.
\end{enumerate}
\section{Conocimientos necesarios y referencias.}
\begin{multicols}{2}
%\setlength{\parindent}{0pt}
\begin{enumerate}[label=\roman*.]
\itemsep-1em 
\item Mediciones directas e indirectas. \\
\item Propagación de incertidumbres. \\
\end{enumerate}
\end{multicols}
\section{Material.}
\begin{multicols}{2}
\begin{enumerate}[label=\alph*.]
\itemsep0em 
\item Vernier.
\item Regla de \SI{30}{\cm}.
\item Vaso de precipitado de $\SI{500}{\milli\liter}$.
\item Juego de pesas.
\item Balanza.
\end{enumerate}
\end{multicols}
\section{Procedimiento experimental.}
\begin{enumerate}
\item Utilizando el vernier mide el diámetro interno del vaso de precipitado, registra el valor de la medición y su incertidumbre.
\item Coloca agua en el vaso de precipitado de tal manera que alcance la marca de \SI{500}{\milli\liter}.
\item Usando la bayoneta del vernier, mide la altura del agua, registra la medición y su incertidumbre.
\item Cada una de las masas tiene anotado en la parte superior el correspondiente peso, anota cada uno de los valores.
\item Registra el valor de cada una de las pesas.
\item Usando la regla de \SI{30}{\cm} mide el ancho y largo de la mesa de trabajo, registra los valores y su incertidumbre.
\end{enumerate}
\section{Análisis experimental.}
\begin{enumerate}
\item Calcula el volumen de agua contenida en el vaso de precipitado a partir del producto del área del vaso y de la altura del agua. Reporta el valor con su incertidumbre.
\item Considerando que la marca del nivel del vaso de precipitado es un valor exacto, ¿qué tanto diferencia hay con el valor de volumen que calculaste? Utiliza un valor porcentual.
\item Para cada una de las pesas reporta la diferencia del valor obtenido, con respecto al valor exacto de cada una de ellas. Nuevamente utiliza un valor porcentual de diferencia.
\item ¿Cuál es el largo y ancho de la mesa? Reporta el valor con su incertidumbre.
\item ¿Identificas alguna fuente de error en las mediciones? ¿Se pueden corregir?  
\end{enumerate}
\vfill
\small{Material elaborado por: M. en C. Gustavo Contreras Mayén. \\ \hspace*{4cm} M. en C. Abraham Lima Buendía.}
\end{document}